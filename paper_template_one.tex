\documentclass[10pt, sigplan, preprint]{acmart}

%% Rights management information.  This information is sent to you
%% when you complete the rights form.  These commands have SAMPLE
%% values in them; it is your responsibility as an author to replace
%% the commands and values with those provided to you when you
%% complete the rights form.
%\setcopyright{acmcopyright}
%\copyrightyear{2020}
%\acmYear{2020}
%\acmDOI{10.1145/TODO}


%\acmJournal{PACMPL}
% \acmVolume{37}
% \acmNumber{4}
% \acmArticle{111}
% \acmMonth{8}

\newenvironment{knitrout}{}{} % an empty environment to be redefined in TeX
\usepackage{url}
\usepackage{amsmath}
\usepackage{listings}
\usepackage{graphicx}
\usepackage{subcaption}
\usepackage{multirow}
\usepackage{tabularx}
\usepackage{hhline}
\usepackage{lscape}
\setcopyright{none}

\RequirePackage{xspace}

\definecolor{gray}{rgb}{0.83, 0.83, 0.83}
\definecolor{blue1dark}{RGB}{137, 175, 188}
\definecolor{blue1medium}{RGB}{182, 214, 228}
\definecolor{blue1light}{RGB}{224, 255, 255}

\usepackage[utf8]{inputenc}
\usepackage{titlesec}
\usepackage[T1]{fontenc}

\input{parsed_one_bench_tables}

\begin{document}

\title{Ruby applications - Analysis of call-site behaviour}

\author{Sophie Kaleba}
\email{s.kaleba@kent.ac.uk}

\maketitle

%!TEX root = paper.tex

\section{FOCUSING ON STANDARD METHOD CALLS}

\section{General characteristics}

\begin{table}[!h]
	\caption{General deata about the benchamrks}
	\centering
	\tiny
	\Metrics
	\label{tab:metrics}
	\end{table}

\section{Call-site behaviour}

Table \ref{tab:metrics} shows the impact of call-site optimisations on the size of the biggest lookup cache.
Each column represents the size of the biggest lookup cache, respectively before any optimisation, after eliminating duplicates ("TP"), and after applying splitting ("SPLIT").

\begin{table}[!h]
	\caption{Receiver distribution}
	\centering
	\tiny
	\Distribution
	\label{tab:distribution}
\end{table}

\subsection{Impact of eliminating target duplication}

Appendix \ref{aftertp} lists the remaining polymorphic call-sites after this optimisation has been applied

\subsubsection{\textbf{Impact on calls}}

Amount of monomorphic, polymorphic and megamorphic calls before eliminating target duplication.
Change. represents the change (in \%) of the number of calls per category, after eliminating target duplication.

\begin{table}[!h]
	\centering
	\tiny
	\AfterTPCalls
	\caption{Eliminating target duplication: impact on calls}
\end{table}

\subsubsection{\textbf{Impact on call sites}}

Amount of monomopphic, polymorphic and megamorphic call-sites before eliminating target duplication.
Change. represents the change (in \%) of the number of call-sites per category, after eliminating target duplication.

\begin{table}[h!]
	\centering
	\tiny
	\AfterTPSites
	\caption{Eliminating target duplication: impact on call-sites}
\end{table}

\subsection{Impact of splitting}
After having eliminated target duplication
Appendix \ref{aftersplitting} lists the remaining polymorphic call-sites after this optimisation has been applied

\subsubsection{\textbf{Impact on calls}}

Amount of monomorphic, polymorphic and megamorphic calls before splitting, but after having eliminated target duplicates.
Change. represents the change (in \%) of the number of calls per category, after applying splitting.

\begin{table}[h!]
	\centering
	\tiny
	\AfterSplitCalls
	\caption{Splitting: impact on calls}
\end{table}

\subsubsection{\textbf{Impact on call sites}}

Table \ref{after_split_call_site} represents the amount of monomorphic, polymorphic and megamorphic call-sites before splitting, but after having eliminated target duplicates.
Change. represents the change (in \%) of the number of call-sites per category, after applying splitting.

\begin{table}[h!]
	\centering
	\tiny
	\AfterSplitSites
	\caption{Splitting: impact on call-sites}
	\label{after_split_call_site}
\end{table}

\section{Detailed summary of splitting behaviour}

Table \ref{transitions} displays the different splitting transitions that occurred. They mention the state of the lookup cache before and after splitting. The number corresponds to the number of splits.

\begin{table}[h!]
	\centering
	\tiny
	\SplittingTransitions
	\caption{The different splitting transitions}
	\label{transitions}
\end{table}

\section{FOCUSING ON BLOCKS}

\section{Call-site behaviour}

\begin{table}[!h]
	\caption{Receiver distribution for blocks}
	\centering
	\tiny
	\BlockDistribution
	\label{tab:blockdistribution}
\end{table}

Following table \ref{tab:blockextent} shows the impact of call-site optimisations on the size of the biggest lookup cache.
Each column represents the size of the biggest lookup cache, respectively before any optimisation and after applying splitting ("SPLIT").

\begin{table}[h!]
	\centering
	\tiny
	\BlockExtent
	\caption{[Blocks] Impact of call-site optimisations on the maximum number of different targets in cache}
	\label{tab:blockextent}
\end{table}

\subsection{Impact of splitting}

\subsubsection{\textbf{Impact on calls}}

Amount of monomorphic, polymorphic and megamorphic calls before splitting, but after having eliminated target duplicates.
Change. represents the change (in \%) of the number of calls per category, after applying splitting.

\begin{table}[h!]
	\centering
	\tiny
	\BlockAfterSplitCallsMini
	\caption{[Blocks] Splitting: impact on calls}
\end{table}

\subsubsection{\textbf{Impact on call sites}}

Table \ref{blocks_after_split_call_site} represents the amount of monomorphic, polymorphic and megamorphic call-sites before splitting, but after having eliminated target duplicates.
Change. represents the change (in \%) of the number of call-sites per category, after applying splitting.

\begin{table}[h!]
	\centering
	\tiny
	\BlockAfterSplitSites
	\caption{[Blocks] Splitting: impact on call-sites}
	\label{blocks_after_split_call_site}
\end{table}

\section{Detailed summary of splitting behaviour}

Table \ref{blocks_transitions} displays the different splitting transitions that occurred. They mention the state of the lookup cache before and after splitting. The number corresponds to the number of splits.

\begin{table}[h!]
	\centering
	\tiny
	\BlockSplittingTransitions
	\caption{[Blocks] The different splitting transitions}
	\label{blocks_transitions}
\end{table}

\section{Caller polymorphism}

Table \ref{caller_poly} displays the number of call-sites a method is called from. We can see that most methods (usually more than half of all methods called) are called from a single call-site.

\begin{landscape}
\begin{table}[h!]
	\centering
	\tiny
	\resizebox{1.5\textwidth}{!}{
	\CallerPolymorphism
	}
	\caption{Caller polymorphism}
	\label{caller_poly}
\end{table}
\end{landscape}

\end{document}
\endinput

