\documentclass[preprint]{acmart}

%% Rights management information.  This information is sent to you
%% when you complete the rights form.  These commands have SAMPLE
%% values in them; it is your responsibility as an author to replace
%% the commands and values with those provided to you when you
%% complete the rights form.
%\setcopyright{acmcopyright}
%\copyrightyear{2020}
%\acmYear{2020}
%\acmDOI{10.1145/TODO}


%\acmJournal{PACMPL}
% \acmVolume{37}
% \acmNumber{4}
% \acmArticle{111}
% \acmMonth{8}

\newenvironment{knitrout}{}{} % an empty environment to be redefined in TeX
\usepackage{tikz}
\usepackage{xspace}
\usepackage{lscape}
\usepackage{framed}
\usepackage{geometry}
\usepackage{longtable}
\usepackage{url}
\setcopyright{none}

\geometry{legalpaper, margin=1in}
\tikzstyle{every picture}+=[font=\sffamily]

\newcommand{\refext}[1]{\detokenize{#1}}

\input{withstartup_method_tables}
\let\StartupGeneralMetrics\GeneralMetrics
\let\StartupBeforeAfterTPCalls\BeforeAfterTPCalls
\let\StartupBeforeAfterTPSites\BeforeAfterTPSites
\let\StartupBeforeAfterSplitCalls\BeforeAfterSplitCalls
\let\StartupBeforeAfterSplitSites\BeforeAfterSplitSites
\let\StartupExtent\Extent
\let\StartupSplittingTransitions\SplittingTransitions
\let\StartupPolyAfterTP\PolyAfterTP
\let\StartupPolyAfterSplitting\PolyAfterSplitting
\input{method_tables}
\let\NoStartupGeneralMetrics\GeneralMetrics
\let\NoStartupBeforeAfterTPCalls\BeforeAfterTPCalls
\let\NoStartupBeforeAfterTPSites\BeforeAfterTPSites
\let\NoStartupBeforeAfterSplitCalls\BeforeAfterSplitCalls
\let\NoStartupBeforeAfterSplitSites\BeforeAfterSplitSites
\let\NoStartupExtent\Extent
\let\NoStartupSplittingTransitions\SplittingTransitions
\let\NoStartupPolyAfterTP\PolyAfterTP
\let\NoStartupPolyAfterSplitting\PolyAfterSplitting


\begin{document}

\title{Ruby-on-rails applications - Analysis of call-site behaviour}

\author{Sophie Kaleba}
\email{s.kaleba@kent.ac.uk}

\maketitle

%!TEX root = paper.tex

\section{FOCUSING ON STANDARD METHOD CALLS}

\section{General characteristics}

\begin{table}[h!]
	\centering
	\resizebox{\linewidth}{!}{
	\StartupGeneralMetrics
	}
	\caption{General data about the benchmark}
\end{table}

\section{Call-site behaviour}

This table shows the impact of call-site optimisations on the size of the biggest lookup cache.
Each column represents the size of the biggest lookup cache, respectively before any optimisation, after eliminating duplicates ("TP"), and after applying splitting ("SPLIT").

\begin{table}[h!]
	\centering
	\StartupExtent
	\caption{Impact of call-site optimisations on the maximum number of different targets in cache}
\end{table}

\subsection{Impact of eliminating target duplication}

Appendix \ref{aftertp} lists the remaining polymorphic call-sites after this optimisation has been applied

\subsubsection{\textbf{Impact on calls}}

Amount of monomorphic, polymorphic and megamorphic calls before eliminating target duplication.
Change. represents the change (in \%) of the number of calls per category, after eliminating target duplication.

\begin{table}[!h]
	\centering
	\StartupBeforeAfterTPCalls
	\caption{Eliminating target duplication: impact on calls}
\end{table}

\subsubsection{\textbf{Impact on call sites}}

Amount of monomopphic, polymorphic and megamorphic call-sites before eliminating target duplication.
Change. represents the change (in \%) of the number of call-sites per category, after eliminating target duplication.

\begin{table}[h!]
	\centering
	\StartupBeforeAfterTPSites
	\caption{Eliminating target duplication: impact on call-sites}
\end{table}

\subsection{Impact of splitting}
After having eliminated target duplication
Appendix \ref{aftersplitting} lists the remaining polymorphic call-sites after this optimisation has been applied

\subsubsection{\textbf{Impact on calls}}

Amount of monomorphic, polymorphic and megamorphic calls before splitting, but after having eliminated target duplicates.
Change. represents the change (in \%) of the number of calls per category, after applying splitting.

\begin{table}[h!]
	\centering	
	\StartupBeforeAfterSplitCalls
	\caption{Splitting: impact on calls}
\end{table}

\subsubsection{\textbf{Impact on call sites}}

Table \ref{after_split_call_site} represents the amount of monomorphic, polymorphic and megamorphic call-sites before splitting, but after having eliminated target duplicates.
Change. represents the change (in \%) of the number of call-sites per category, after applying splitting.

\begin{table}[h!]
	\centering
	\StartupBeforeAfterSplitSites
	\caption{Splitting: impact on call-sites}
	\label{after_split_call_site}
\end{table}

\section{Detailed summary of splitting behaviour}

Table \ref{transitions} displays the different splitting transitions that occurred. They mention the state of the lookup cache before and after splitting. The number corresponds to the number of splits.

\begin{table}[h!]
	\centering
	\StartupSplittingTransitions
	\caption{The different splitting transitions}
	\label{transitions}
\end{table}

\section{Startup behaviour}
Startup here is the part of execution time where user code is not yet loaded, and contains RubyContext initialisation, including the loading of Ruby core library. In this section, \textbf{we use the same tables as previously, but calls that have happened in the startup phase have been discarded from the data.}

First, let's see how much this influences the degree of polymorphism.

\begin{table}[h!]
	\centering
	\NoStartupExtent
	\caption{[Startup discarded] Impact of call-site optimisations on the maximum number of different targets in cache}
	\label{transitions}
\end{table}


\subsection{Impact of eliminating target duplication}
\subsubsection{\textbf{Impact on calls}}

Amount of monomorphic, polymorphic and megamorphic calls before eliminating target duplication.
Change. represents the change (in \%) of the number of calls per category, after eliminating target duplication.

\begin{table}[!h]
	\centering
	\NoStartupBeforeAfterTPCalls
	\caption{[Startup discarded] Eliminating target duplication: impact on calls}
\end{table}

\subsubsection{\textbf{Impact on call sites}}

Amount of monomopphic, polymorphic and megamorphic call-sites before eliminating target duplication.
Change. represents the change (in \%) of the number of call-sites per category, after eliminating target duplication.

\begin{table}[h!]
	\centering
	\NoStartupBeforeAfterTPSites
	\caption{[Startup discarded] Eliminating target duplication: impact on call-sites}
\end{table}

\subsection{Impact of splitting}
After having eliminated target duplication
Appendix \ref{aftersplitting} lists the remaining polymorphic call-sites after this optimisation has been applied

\subsubsection{\textbf{Impact on calls}}

Amount of monomorphic, polymorphic and megamorphic calls before splitting, but after having eliminated target duplicates.
Change. represents the change (in \%) of the number of calls per category, after applying splitting.

\begin{table}[h!]
	\centering	
	\NoStartupBeforeAfterSplitCalls
	\caption{[Startup discarded] Splitting: impact on calls}
\end{table}

\subsubsection{\textbf{Impact on call sites}}

Table \ref{after_split_call_site} represents the amount of monomorphic, polymorphic and megamorphic call-sites before splitting, but after having eliminated target duplicates.
Change. represents the change (in \%) of the number of call-sites per category, after applying splitting.

\begin{table}[h!]
	\centering
	\NoStartupBeforeAfterSplitSites
	\caption{[Startup discarded] Splitting: impact on call-sites}
	\label{after_split_call_site}
\end{table}

\section{Detailed summary of splitting behaviour}

Table \ref{transitions} displays the different splitting transitions that occurred. They mention the state of the lookup cache before and after splitting. The number corresponds to the number of splits.

\begin{table}[h!]
	\centering
	\NoStartupSplittingTransitions
	\caption{[Startup discarded] The different splitting transitions}
	\label{transitions}
\end{table}


\clearpage
\appendix
\section{After eliminating target duplicates: focus on the remaining polymorphic call-sites}
\label{aftertp}

This table lists the remaining polymorphic call-sites, after eliminating target duplicates

\begin{table}[h!]
	\centering
	\resizebox{\linewidth}{!}{
	\PolyAfterTP
	}
	\caption{List of polymorphic call-sites after eliminating target duplicates}
\end{table}


\section{After splitting: focus on the remaining polymorphic call-sites}
\label{aftersplitting}

This table lists the remaining polymorphic call-sites, after splitting

\begin{table}[h!]
	\centering
	\resizebox{\linewidth}{!}{
	\PolyAfterSplitting
	}
	\caption{List of polymorphic call-sites after splitting}
\end{table}

\end{document}
\endinput
%%
%% End of file `sample-acmsmall.tex'.

