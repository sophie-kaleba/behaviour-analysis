\documentclass[preprint]{acmart}

%% Rights management information.  This information is sent to you
%% when you complete the rights form.  These commands have SAMPLE
%% values in them; it is your responsibility as an author to replace
%% the commands and values with those provided to you when you
%% complete the rights form.
%\setcopyright{acmcopyright}
%\copyrightyear{2020}
%\acmYear{2020}
%\acmDOI{10.1145/TODO}


%\acmJournal{PACMPL}
% \acmVolume{37}
% \acmNumber{4}
% \acmArticle{111}
% \acmMonth{8}

\newenvironment{knitrout}{}{} % an empty environment to be redefined in TeX
\usepackage{tikz}
\usepackage{xspace}
\usepackage{lscape}
\usepackage{framed}
\usepackage{geometry}
\usepackage{longtable}
\usepackage{url}
\setcopyright{none}

\geometry{legalpaper, margin=1in}
\tikzstyle{every picture}+=[font=\sffamily]

\newcommand{\refext}[1]{\detokenize{#1}}

\input{gen-eval}


\begin{document}

\title{Ruby-on-rails applications - Analysis on call-site behaviour}

\author{Sophie Kaleba}
\email{s.kaleba@kent.ac.uk}

\maketitle

\section{Introduction}

These are results are about the \BenchmarkName benchmark, that has been run for \NumIterations iterations, and \NumInnerIterations inner iterations, on a \Hihaho.
The raw log file is located at $\LogFilename$. 

\section{Global behaviour}

In this experiment, \NumberCallSites call-sites were executed, totaling \NumberCalls calls.  
The following table lists the thirty most executed call-sites within this execution\footnote{This does not account for the actual execution time spent}.

\MostExecutedCallSite

\section{Call-site behaviours}

\subsection{Splitting}

Within this execution, \NumOfSplittedTargets call-sites have been splitted, totalling \NumsOfSplits
Some call-targets have been splitted several times; the following table list all the call targets that have been splitted.

\newpage

\begin{landscape}
\begin{figure}[h]
  \centering
  \Splitting
  \caption{
    30th most splitted targets}
\end{figure}
\end{landscape}

\begin{figure}[h]
  \centering
  \EffectSplittingSites
  \caption{
    Effect of splitting on lookup-cache status (number of sites impacted)}
\end{figure}

\begin{figure}[h]
  \centering
  \EffectSplittingFrequency
  \caption{
    Effect of splitting on lookup-cache status (on all splits)}
\end{figure}


\subsection{Type of lookup caches}

\subsubsection{Receiver polymorphism}

\begin{figure}[h!]
  \centering
  \CacheDistribObservedSplitting
  \caption{
    Observed cache distribution}
\end{figure}


\begin{figure}[h!]
  \centering
  \CacheDistribOriginalSplitting
  \caption{
    Original cache distribution}
\end{figure}


\subsubsection{Target polymorphism}

\begin{figure}[h!]
  \centering
  \TargetPolymorphism
  \caption{
    Impact of optimization regarding target polymorphism}
\end{figure}

\begin{landscape}
\begin{figure}[h]
  \centering
  \TargetPolymorphismDetails
  \caption{
    Detailed impact with the evolution of cache entries}
\end{figure}
\end{landscape}


%\subsection{Type of call-sites as per ruby semantics}
%
%%There are XXXX different types of calls in Ruby:
%%\begin{itemize}
%%\item 
%%\item 
%%\item 
%%\end{itemize}
%
%The following table displays basic summary data based on the type of ruby call-site.



\section{Identifying patterns in execution}

The call-sites qualifying for phase-based splitting have to fit within a set of criterion; namely, they have to:
 \begin{itemize}
 \item be polymorphic or megamorphic
 \item be called a significant amount of times at runtime
 \end{itemize}

A manual inspection is then required to identify potential phases based on the lookup cache type. The following table lists the first forty-five most executed polymorphic call-sites.

\begin{landscape}

\SuperHottestMegamorphic{}

\end{landscape}

\appendix

\bibliographystyle{ACM-Reference-Format}
\bibliography{sample-base}


\end{document}
\endinput
%%
%% End of file `sample-acmsmall.tex'.

